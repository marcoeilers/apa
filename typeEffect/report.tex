\documentclass[a4paper,11pt]{article}
\usepackage[utf8]{inputenc}
\usepackage[T1]{fontenc}

\headsep1cm
\parindent0cm
\usepackage{amssymb, amstext, amsmath}
\usepackage{fancyhdr}
\usepackage{lastpage}
\usepackage{graphicx}
\usepackage{bussproofs}
%\usepackage{fullpage}
\usepackage{tikz}
\usepackage{listings}
\usepackage{mathtools}

\usepackage{hyperref}  
\newcommand{\wpe}[2]{\texttt{wp }#1\texttt{ }#2}
\newcommand\numberthis{\addtocounter{equation}{1}\tag{\theequation}}

\headsep1cm


\lstset{basicstyle=\footnotesize}
\usetikzlibrary{arrows,automata}

\lhead{\textbf{Project 2: Type and Effect Systems}}
\rhead{}

\cfoot{}
\lfoot{Marco Eilers, Bas in het Veld}
\rfoot{\thepage\ of \pageref{LastPage}}
\pagestyle{fancy}
\renewcommand{\footrulewidth}{0.4pt}

\setlength{\parskip}{4pt}

\begin{document}

\title{Automatic Program Analysis\\Project 2: Type and Effect Systems}
\author{Marco Eilers - F121763, Bas in het Veld - 3710971}

\maketitle

\section{Language to be Analyzed}

For our assignment, we decided to use the \emph{FUN} language, as described in \emph{Nielson, Nielson \& Hankin}, chapter 5.1.
The \emph{FUN} language is an untyped lambda calculus, with operators for recursion, and conditional and local definitions.
Program fragments with labels are \emph{expressions}, and program fragments without labels are \emph{terms}.
Below is a general overview of the syntax of this language:
\\
\\
\begin{tabular}{l l}
Expressions: &	$e \in \textbf{Exp}$ \\
Terms: &		$t \in \textbf{Term}$ \\
 & \\
Variables: &		$f,x \in \textbf{Var}$ \\
Constants &		$c \in \textbf{Const} $ \\
Binary operators &	$op \in \textbf{Op} $ \\
Labels &		$l \in \textbf{Lab} $ \\
\end{tabular}
\\
\\
The \emph{abstract syntax} is:
\\
\\
\begin{tabular}{l}
$e ::= t^l$ \\
$t ::= c\:|\:x\:|\:\text{fn}\:x\:=>\:e_0\:|\:\text{fun}\:f\:x\:=>\:e_0\:|\:e_1\:e_2\:|$ \\
$\:\:\:\text{if}\:e_0\:\text{then}\:e_1\:\text{else}\:e_2\:|\:\text{let}\:x\:=\:e_1\:\text{in}\:e_2\:|\:e_1\:\text{op}\:e_2$ \\
\end{tabular}
\\
\\
In this syntax, there are two distinct function definitions: $\text{fn}\:x\:=>\:e_0$ is a normal function definition, and $\text{fun}\:f\:x\:=>\:e_0$ is its recursive variant.
In the latter, any free occurrence of $f$ in $e_0$ refers to $\text{fun}\:f\:x\:=>\:e_0$.
Finally, $\text{let}\:x\:=\:e_1\:\text{in}\:e_2$ is a local definition.

\section{Control Flow Analysis}
\subsection{Annotated Type System}
\begin{prooftree}
\AxiomC{$\hat{\Gamma} \vdash_{CFA} t_1 : \hat{\sigma_1}$}
\AxiomC{$\hat{\Gamma}[x \mapsto \hat{\sigma_1}] \vdash_{CFA} t_2 : \hat{\tau}$}
\BinaryInfC{$\hat{\Gamma} \vdash_{CFA} \textbf{let }x = t_1 \textbf{ in } t_2 : \hat{\tau}$}
\end{prooftree}

\begin{prooftree}
\AxiomC{$\hat{\Gamma} \vdash_{CFA} t_1 : \tau_\oplus^1 $}
\AxiomC{$\hat{\Gamma} \vdash_{CFA} t_2 : \tau_\oplus^2 $}
\BinaryInfC{$\hat{\Gamma} \vdash_{CFA} t_1 \oplus t_2 : \tau_\oplus$}
\end{prooftree}

\begin{prooftree}
\AxiomC{$\hat{\Gamma} \vdash_{CFA} t_0 : \textbf{ } Bool$}
\AxiomC{$\hat{\Gamma} \vdash_{CFA} t_1 : \hat{\tau}$}
\AxiomC{$\hat{\Gamma} \vdash_{CFA} t_2 : \hat{\tau}$}
\TrinaryInfC{$\hat{\Gamma} \vdash_{CFA} \textbf{if } t_0 \textbf{ then }t_1 \textbf{ else } t_2 : \hat{\tau}$}
\end{prooftree}

\begin{prooftree}
\AxiomC{$\hat{\Gamma} \vdash_{CFA} t_1 : \hat{\tau_2} \xrightarrow{\varphi} \hat{\tau}$}
\AxiomC{$\hat{\Gamma} \vdash_{CFA} t_2 : \hat{\tau_2}$}
\BinaryInfC{$\hat{\Gamma} \vdash_{CFA} t_1 t_2 : \hat{\tau}$}
\end{prooftree}

\begin{prooftree}
\AxiomC{$\hat{\Gamma}[x \mapsto \hat{\tau_1}] \vdash_{CFA} t_1 : \hat{\tau_2}$}
\UnaryInfC{$\hat{\Gamma} \vdash_{CFA} \textbf{fn}_\pi \texttt{ }x => (t_1) : \hat{\tau_1} \xrightarrow{\{\pi\}}\hat{\tau_2}$}
\end{prooftree}

\begin{prooftree}
\AxiomC{$\hat{\Gamma}[f \mapsto (\hat{\tau_1} \xrightarrow{\{\pi\}}\hat{\tau_2})][x \mapsto \hat{\tau_1}] \vdash_{CFA} t_1 : \hat{\tau_2}$}
\UnaryInfC{$\hat{\Gamma} \vdash_{CFA} \textbf{fun}_\pi \textbf{ }f \textbf{ }x => (t_1) : \hat{\tau_1} \xrightarrow{\{\pi\}}\hat{\tau_2}$}
\end{prooftree}

\begin{prooftree}
\AxiomC{$\hat{\Gamma}(x) = \hat{\sigma}$}
\UnaryInfC{$\hat{\Gamma} \vdash_{CFA} x : \hat{\sigma}$}
\end{prooftree}

\begin{prooftree}
\AxiomC{}
\UnaryInfC{$\hat{\Gamma} \vdash_{CFA} n : Int$}
\end{prooftree}

\begin{prooftree}
\AxiomC{}
\UnaryInfC{$\hat{\Gamma} \vdash_{CFA} \textbf{true} : Bool$}
\end{prooftree}

\begin{prooftree}
\AxiomC{}
\UnaryInfC{$\hat{\Gamma} \vdash_{CFA} \textbf{false} : Bool$}
\end{prooftree}

\begin{prooftree}
\AxiomC{$\hat{\Gamma} \vdash_{CFA} t_1 : \hat{\tau_1}$}
\AxiomC{$\hat{\Gamma} \vdash_{CFA} t_2 : \hat{\tau_2}$}
\BinaryInfC{$\hat{\Gamma} \vdash_{CFA} \textbf{Pair}_\pi(t_1, t_2) : (\hat{\tau_1} \times^{\{\pi\}} \hat{\tau_2})$}
\end{prooftree}

\begin{prooftree}
\AxiomC{$\hat{\Gamma} \vdash_{CFA} t_1 : \hat{\tau}$}
\AxiomC{$\hat{\Gamma} \vdash_{CFA} t_1 : \hat{\tau} \textbf{ }list^{\varphi}$}
\BinaryInfC{$\hat{\Gamma} \vdash_{CFA} \textbf{Cons}_\pi(t_1, t_2) :  \hat{\tau} \textbf{ }list^{\{\pi\}}$}
\end{prooftree}

\begin{prooftree}
\AxiomC{}
\UnaryInfC{$\hat{\Gamma} \vdash_{CFA} \textbf{Nil}_\pi : \forall \alpha. \alpha \textbf{ } list^{\{\pi\}}$}
\end{prooftree}

\begin{prooftree}
\AxiomC{$\hat{\Gamma} \vdash_{CFA} t_0 : (\hat{\tau_1} \times^\varphi \hat{\tau_2}) $}
\AxiomC{$\hat{\Gamma}[x_1 \mapsto \hat{\tau_1}][x_2 \mapsto \hat{\tau_2}] \vdash_{CFA} t_1 : \hat{\tau}$}
\BinaryInfC{$\hat{\Gamma} \vdash_{CFA} \textbf{case } t_0 \textbf{ of Pair}(x_1, x_2) => t_1 : \hat{\tau}$}
\end{prooftree}

\begin{prooftree}
\AxiomC{$\hat{\Gamma} \vdash_{CFA} t_0 : \hat{\tau_0} \textbf{ }list^\varphi$}
\AxiomC{$\hat{\Gamma}[x_1 \mapsto \hat{\tau_0}][x_2 \mapsto \hat{\tau_0}\textbf{ }list^\varphi ] \vdash_{CFA} t_1 : \hat{\tau}$}
\AxiomC{$\hat{\Gamma} \vdash_{CFA} t_2 : \hat{\tau}$}
\TrinaryInfC{$\hat{\Gamma} \vdash_{CFA} \textbf{case }t_0 \textbf{ of Cons}(x_1, x_2) => t_1 \textbf{ or }t_2 : \hat{\tau}$}
\end{prooftree}

\subsection{Algorithm}
\subsection{Polymorphism}
\subsection{Polyvariance}


\section{Evaluation of our Solution}


\section{Compiling and Running our Code}


\section{Example Programs and Results}


\end{document}
