\documentclass[a4paper,11pt]{article}
\usepackage[utf8]{inputenc}
\usepackage[T1]{fontenc}

\headsep1cm
\parindent0cm
\usepackage{amssymb, amstext, amsmath}
\usepackage{fancyhdr}
\usepackage{lastpage}
\usepackage{graphicx}
\usepackage{bussproofs}
%\usepackage{fullpage}
\usepackage{tikz}
\usepackage{listings}
\usepackage{mathtools}

\usepackage{hyperref}  
\newcommand{\wpe}[2]{\texttt{wp }#1\texttt{ }#2}
\newcommand\numberthis{\addtocounter{equation}{1}\tag{\theequation}}
\newcommand{\bigU}{\ensuremath{\mathcal{U}_{CFA}}}
\newcommand{\bigW}{\ensuremath{\mathcal{W}_{CFA}}}

\headsep1cm


\lstset{basicstyle=\footnotesize}
\usetikzlibrary{arrows,automata}

\lhead{\textbf{Project 2: Type and Effect Systems}}
\rhead{}

\cfoot{}
\lfoot{Marco Eilers, Bas in het Veld}
\rfoot{\thepage\ of \pageref{LastPage}}
\pagestyle{fancy}
\renewcommand{\footrulewidth}{0.4pt}

\setlength{\parskip}{4pt}

\begin{document}

\title{Automatic Program Analysis\\Project 2: Type and Effect Systems}
\author{Marco Eilers - F121763, Bas in het Veld - 3710971}

\maketitle

\section{Introduction}
We implemented a Control Flow Analysis for the \emph{FUN} language (described in the next section), extended with syntax for basic data structures. The analysis tracks the creation points of both functions and data structures, and is both polymorphic and polyvariant. Our implementation is written in Haskell. 

\section{Language to be Analyzed}

For our assignment, we decided to use the \emph{FUN} language, as described in \emph{Nielson, Nielson \& Hankin}, chapter 5.1.
The \emph{FUN} language is an untyped lambda calculus, with operators for recursion, and conditional and local definitions.
Program fragments with labels are \emph{expressions}, and program fragments without labels are \emph{terms}.
Below is a general overview of the syntax of this language:
\\
\\
\begin{tabular}{l l}
Expressions: &	$e \in \textbf{Exp}$ \\
Terms: &		$t \in \textbf{Term}$ \\
 & \\
Variables: &		$f,x \in \textbf{Var}$ \\
Constants &		$c \in \textbf{Const} $ \\
Binary operators &	$\oplus \in \textbf{Op} $ \\
Labels &		$\pi \in \textbf{Lab} $ \\
\end{tabular}
\\
\\
The \emph{abstract syntax} is:
\\
\\
\begin{tabular}{l}
$e ::= t_\pi$ \\
$t ::= c\:|\:x\:|\:\textbf{fn}\:x\:=>\:(e_0)\:|\:\textbf{fun}\:f\:x\:=>\:(e_0)\:|\:e_1\:e_2\:|$ \\
$\:\:\:\textbf{if}\:e_0\:\textbf{then}\:e_1\:\textbf{else}\:e_2\:|\:\textbf{let}\:x\:=\:e_1\:\textbf{in}\:e_2\:|\:e_1\:\oplus\:e_2\:|$\\
$\:\:\:\textbf{Pair}(e_1,\:e_2)\:|\: \textbf{case}\:e_0\:\textbf{of Pair}(x_1,\:x_2)\:=>\:e_1\:|$ \\
$\:\:\:\textbf{Nil}\:|\:\textbf{Cons}(e_1,\:e_2)\:|\:\textbf{case}\:e_0\:\textbf{of Cons}(x_1,\:x_2)\:=>\:e_1\:\textbf{or}\:e_2$ \\
$c ::= n\:|\:\textbf{true}\:|\:\textbf{false}$ \\
$\oplus ::= +\:|\:-\:|\:*\:|\:\&\&\:|\:||\:|\:<\:|\:>$
\end{tabular}
\\
\\
In this syntax, there are two distinct function definitions: $\text{fn}\:x\:=>\:(e_0)$ is a normal function definition, and $\text{fun}\:f\:x\:=>\:(e_0)$ is its recursive variant.
In the latter, any free occurrence of $f$ in $e_0$ refers to $\text{fun}\:f\:x\:=>\:(e_0)$.
Finally, $\text{let}\:x\:=\:e_1\:\text{in}\:e_2$ is a local definition. 

This concludes the original language described in NNH. Since we wanted to include data structures for our analyses, we added five more constructs. For pairs, we have the constructor $\textbf{Pair}(e_1,\:e_2)$, which creates a pair of two expressions, and the case statement $\textbf{case}\:e_0\:\textbf{of Pair}(x_1,\:x_2)\:=>\:e_1$, where $e_0$ must evaluate to a pair and $e_1$ may use the variables $x_1$ and $x_2$. The constructs for lists are similar. We have the constructors \textbf{Nil}, which creates an empty list, and $\textbf{Cons}(e_1,\:e_2)$, which prepends $e_1$ to the list $e_2$. Finally, we have the case statement $\textbf{case}\:e_0\:\textbf{of Cons}(x_1,\:x_2)\:=>\:e_1\:\textbf{or}\:e_2$, where $e_0$ must evaluate to a list, $e_1$ may refer to the variables $x_1$ and $x_2$ and is evaluated if $e_0$ evaluates to a non-empty list, and $e_2$ is evaluated if the list is empty (and may not refer to the variables).

While we were initially planning to include the possibility to add user-defined data structures, we decided to focus on the actual analysis first and finally had no time to add more features to the language.

We wrote our own parser for \emph{FUN}, which can be found in the file \texttt{FunParser.hs}. It works fine for most examples we tried, but since we did not want to spend too much time on the parser, it is not perfect. Therefore it is sometimes necessary to add seemingly superfluous parentheses around expressions (see for example the syntax for functions above). In particular, there must be parentheses around every function application. 

The data structures used for storing programs can be found in the file \texttt{AST.hs}.

\section{Control Flow Analysis}


\subsection{Annotated Type System}
We perform our analysis using an annotated type system that looks like this:

\begin{tabular}{l l}
Annotations: & $\varphi \in \textbf{Ann}$ \\
Annotation Variables: &	$\beta  \in \textbf{AVar}$ \\
Simple Types: &		$\hat{\tau} \in \textbf{SType}$ \\
Type Variables: & $\alpha \in \textbf{TVar}$ \\
Qualified Types & $\hat{\rho} \in \textbf{QualType}$\\
Simple Type Schemes: &		$\hat{\sigma} \in \textbf{TScheme}$ \\
Simple Type Environments &		$\hat{\Gamma} \in \textbf{TEnv} $ \\
Type Substitution & $\hat{\theta} \in \textbf{TSubst}$\\
Constraints &	$C \in \textbf{Constraint}$ \\
\end{tabular}

where

\begin{tabular}{l}
$\varphi ::= \emptyset\:|\:\beta\:|\:\{\pi\}\:|\:\varphi_1 \cup \varphi_2$\\
$\hat{\tau} ::= \alpha\:|\:Nat\:|\:Bool\:|\:(\hat{\tau_1}\xrightarrow{\beta}\hat{\tau_2})\:|\:(\hat{\tau_1} \times^\beta \hat{\tau_2})\:|\:(\hat{\tau_1}\:list^\beta)$ \\
$\hat{\rho} ::= \hat{\tau}\:|\:C\Rightarrow \hat{\tau}$ \\
$\hat{\sigma} ::= \hat{\rho}\:|\:\forall \alpha.\hat{\sigma_1}\:|\:\forall \beta.\hat{\sigma_1}$ \\
$\hat{\Gamma} ::= []\:|\:\hat{\Gamma_1}[x \mapsto \hat{\sigma}]$ \\
$\hat{\theta} ::= []\:|\:\hat{\theta}[\alpha \mapsto \hat{\tau}] $ \\
$C ::= \emptyset\:|\:\{\beta \supseteq \varphi\}\:|\:C_1 \cup C_2$
\end{tabular}

Function, pair and list types contain annotation variables. Their annotations specify the set of labels where a function/pair/list may have been created. For lists, the set contains the set of all labels where parts of the list may have been created. 

The (non syntax directed) inference rules for our type system are the following. For rules that do not use the label of the statement, the label is omitted.

\begin{prooftree}
\AxiomC{}
\RightLabel{[\textsc{T-Int}]}
\UnaryInfC{$\hat{\Gamma} \vdash_{CFA} n : Int$}
\end{prooftree}

\begin{prooftree}
\AxiomC{}
\RightLabel{[\textsc{T-True}]}
\UnaryInfC{$\hat{\Gamma} \vdash_{CFA} \textbf{true} : Bool$}
\end{prooftree}

\begin{prooftree}
\AxiomC{}
\RightLabel{[\textsc{T-False}]}
\UnaryInfC{$\hat{\Gamma} \vdash_{CFA} \textbf{false} : Bool$}
\end{prooftree}

\begin{prooftree}
\AxiomC{$\hat{\Gamma}(x) = \hat{\sigma}$}
\RightLabel{[\textsc{T-Var}]}
\UnaryInfC{$\hat{\Gamma} \vdash_{CFA} x : \hat{\sigma}$}
\end{prooftree}

\begin{prooftree}
\AxiomC{$\hat{\Gamma}[x \mapsto \hat{\tau_1}] \vdash_{CFA} e_1 : \hat{\tau_2}$}
\RightLabel{[\textsc{T-Fn}]}
\UnaryInfC{$\hat{\Gamma} \vdash_{CFA} \textbf{fn}_\pi \texttt{ }x => (e_1) : \hat{\tau_1} \xrightarrow{\{\pi\}}\hat{\tau_2}$}
\end{prooftree}

\begin{prooftree}
\AxiomC{$\hat{\Gamma}[f \mapsto (\hat{\tau_1} \xrightarrow{\{\pi\}}\hat{\tau_2})][x \mapsto \hat{\tau_1}] \vdash_{CFA} e_1 : \hat{\tau_2}$}
\RightLabel{[\textsc{T-Fun}]}
\UnaryInfC{$\hat{\Gamma} \vdash_{CFA} \textbf{fun}_\pi \textbf{ }f \textbf{ }x => (e_1) : \hat{\tau_1} \xrightarrow{\{\pi\}}\hat{\tau_2}$}
\end{prooftree}

\begin{prooftree}
\AxiomC{$\hat{\Gamma} \vdash_{CFA} e_1 : \hat{\tau_2} \xrightarrow{\varphi} \hat{\tau}$}
\AxiomC{$\hat{\Gamma} \vdash_{CFA} e_2 : \hat{\tau_2}$}
\RightLabel{[\textsc{T-App}]}
\BinaryInfC{$\hat{\Gamma} \vdash_{CFA} e_1 e_2 : \hat{\tau}$}
\end{prooftree}

\begin{prooftree}
\AxiomC{$\hat{\Gamma} \vdash_{CFA} e_1 : \hat{\sigma_1}$}
\AxiomC{$\hat{\Gamma}[x \mapsto \hat{\sigma_1}] \vdash_{CFA} e_2 : \hat{\tau}$}
\RightLabel{[\textsc{T-Let}]}
\BinaryInfC{$\hat{\Gamma} \vdash_{CFA} \textbf{let }x = e_1 \textbf{ in } e_2 : \hat{\tau}$}
\end{prooftree}

\begin{prooftree}
\AxiomC{$\hat{\Gamma} \vdash_{CFA} e_1 : \tau_\oplus^1 $}
\AxiomC{$\hat{\Gamma} \vdash_{CFA} e_2 : \tau_\oplus^2 $}
\RightLabel{[\textsc{T-Op}]}
\BinaryInfC{$\hat{\Gamma} \vdash_{CFA} e_1 \oplus e_2 : \tau_\oplus$}
\end{prooftree}

\begin{prooftree}
\AxiomC{$\hat{\Gamma} \vdash_{CFA} e_0 : \textbf{ } Bool$}
\AxiomC{$\hat{\Gamma} \vdash_{CFA} e_1 : \hat{\tau}$}
\AxiomC{$\hat{\Gamma} \vdash_{CFA} e_2 : \hat{\tau}$}
\RightLabel{[\textsc{T-If}]}
\TrinaryInfC{$\hat{\Gamma} \vdash_{CFA} \textbf{if } e_0 \textbf{ then }e_1 \textbf{ else } e_2 : \hat{\tau}$}
\end{prooftree}


\begin{prooftree}
\AxiomC{$\hat{\Gamma} \vdash_{CFA} e_1 : \hat{\tau_1}$}
\AxiomC{$\hat{\Gamma} \vdash_{CFA} e_2 : \hat{\tau_2}$}
\RightLabel{[\textsc{T-Pair}]}
\BinaryInfC{$\hat{\Gamma} \vdash_{CFA} \textbf{Pair}_\pi(e_1, e_2) : (\hat{\tau_1} \times^{\{\pi\}} \hat{\tau_2})$}
\end{prooftree}

\begin{prooftree}
\AxiomC{$\hat{\Gamma} \vdash_{CFA} e_1 : \hat{\tau}$}
\AxiomC{$\hat{\Gamma} \vdash_{CFA} e_1 : \hat{\tau} \textbf{ }list^{\varphi}$}
\RightLabel{[\textsc{T-Cons}]}
\BinaryInfC{$\hat{\Gamma} \vdash_{CFA} \textbf{Cons}_\pi(e_1, e_2) :  \hat{\tau} \textbf{ }list^{\{\pi\} \cup \varphi}$}
\end{prooftree}

\begin{prooftree}
\AxiomC{}
\RightLabel{[\textsc{T-Nil}]}
\UnaryInfC{$\hat{\Gamma} \vdash_{CFA} \textbf{Nil}_\pi : \hat{\tau} \textbf{ } list^{\{\pi\}}$}
\end{prooftree}

\begin{prooftree}
\AxiomC{$\hat{\Gamma} \vdash_{CFA} e_0 : (\hat{\tau_1} \times^\varphi \hat{\tau_2}) $}
\AxiomC{$\hat{\Gamma}[x_1 \mapsto \hat{\tau_1}][x_2 \mapsto \hat{\tau_2}] \vdash_{CFA} e_1 : \hat{\tau}$}
\RightLabel{[\textsc{T-Case-Pair}]}
\BinaryInfC{$\hat{\Gamma} \vdash_{CFA} \textbf{case } e_0 \textbf{ of Pair}(x_1, x_2) => e_1 : \hat{\tau}$}
\end{prooftree}

\begin{prooftree}
\AxiomC{$\hat{\Gamma} \vdash_{CFA} e_0 : \hat{\tau_0} \textbf{ }list^\varphi$}
\AxiomC{$\hat{\Gamma}[x_1 \mapsto \hat{\tau_0}][x_2 \mapsto \hat{\tau_0}\textbf{ }list^\varphi ] \vdash_{CFA} e_1 : \hat{\tau}$}
\AxiomC{$\hat{\Gamma} \vdash_{CFA} e_2 : \hat{\tau}$}
\RightLabel{[\textsc{T-Case-List}]}
\TrinaryInfC{$\hat{\Gamma} \vdash_{CFA} \textbf{case }e_0 \textbf{ of Cons}(x_1, x_2) => e_1 \textbf{ or }e_2 : \hat{\tau}$}
\end{prooftree}

%gen
\begin{prooftree}
\AxiomC{}
\RightLabel{[\textsc{T-Gen}]}
\UnaryInfC{}
\end{prooftree}

%ann gen
\begin{prooftree}
\AxiomC{}
\RightLabel{[\textsc{T-Gen-Ann}]}
\UnaryInfC{}
\end{prooftree}


\begin{prooftree}
\AxiomC{$\hat{\Gamma} \vdash_{CFA} e : \forall \alpha. \hat{\sigma}$}
\RightLabel{[\textsc{T-Inst}]}
\UnaryInfC{$\hat{\Gamma} \vdash_{CFA} e : \hat{\sigma}[\hat{\tau} / \alpha]$}
\end{prooftree}


\begin{prooftree}
\AxiomC{$\hat{\Gamma} \vdash_{CFA} e : \forall \beta. \hat{\sigma}$}
\RightLabel{[\textsc{T-Inst-Ann}]}
\UnaryInfC{$\hat{\Gamma} \vdash_{CFA} e : \hat{\sigma}[\varphi_1 / \beta]$}
\end{prooftree}

\begin{prooftree}
\AxiomC{$\hat{\Gamma} \vdash_{CFA} e : \tau^\varphi$}
\RightLabel{[\textsc{T-Sub}]}
\UnaryInfC{$\hat{\Gamma} \vdash_{CFA} e : \tau^{\varphi \cup \varphi'}$}
\end{prooftree}

\begin{prooftree}
\AxiomC{}
\RightLabel{[\textsc{T-Qual}]}
\UnaryInfC{}
\end{prooftree}

\begin{prooftree}
\AxiomC{}
\RightLabel{[\textsc{T-Res}]}
\UnaryInfC{}
\end{prooftree}

Most of the rules have the same or almost the same form as discussed in the lectures and in NNH. 

The rules for the new data types are relatively straightforward. To get the type of a pair, we use the types of its elements and add the current statement's label as the annotation. 
For the case statement for pairs, if the first argument evaluates to a pair, the resulting type is the type of the second argument in the current environment with the given variable names pointing to the types of the elements of the pair.

\textbf{Nil} has any list type with the annotation being the current label. For \textbf{Cons}, if the type of the second argument is a list of the type of the first argument, we get the type of the second argument, with the annotation being the union of the current label and the annotation resulting from the second argument. 

The case statement for lists is the most complicated one: If the first argument evaluates to a list, and the second and third argument evaluate to the same type (the third in the current environment, the second in the current environment with the first given variable pointing to the type of the first element of the list, the second to the type of the list itself), then the entire case statement will also evaluate to this type.

In addition to that TODO

\subsection{Algorithm}
Our algorithm is a modified version of Algorithm W. We adapted the algorithm from NNH (p. 306 ff) to achieve polymorphism, polyvariance and better subeffecting. 

The algorithm centers around the functions $\mathcal{U}_{CFA}$ and $\mathcal{W}_{CFA}$. Their types are the following:

\begin{tabular}{l l l l}
\bigU & : & $\textbf{SType} \rightarrow  \textbf{SType} \rightarrow (\textbf{TSubst} \times \textbf{Constraint}) $\\
\bigW & : & $\textbf{TEnv} \rightarrow \textbf{SType} \rightarrow (\textbf{SType} \times \textbf{TSubst} \times \textbf{Constraint})$\\
generalise & : & $\textbf{TEnv} \rightarrow \textbf{SType} \rightarrow \textbf{SAnn} \rightarrow \textbf{Constraint} \rightarrow \textbf{TScheme}$\\
instantiate & : & $\textbf{TScheme} \rightarrow (\textbf{SType} \times \textbf{Constraint})$
\end{tabular}

The function \bigU takes two types $\hat{\tau_1}$ and $\hat{\tau_2}$ as inputs and generates a substitution $\theta$ such that $\theta\hat{\tau_1} = \theta \hat{\tau_2}$, and a constraint $C$ such that, for every annotation variable $\beta_1$ in $\hat{\tau_1}$ and the corresponding variable $\beta_2$ in $\hat{\tau_2}$, $\beta_2 \supseteq \beta_1$; i.e. $\beta_1$ flows into $\beta_2$. It is defined as follows:

\begin{align*}
\bigU (Nat, Nat) &= ([], \emptyset) \\
\bigU (Bool, Bool) &= ([], \emptyset) \\
\bigU (\hat{\tau_1} \xrightarrow{\beta_1} \hat{\tau_2}, \hat{\tau_3} \xrightarrow{\beta_2} \hat{\tau_4}) &= (\theta_1 \circ \theta_2, \{\beta_2 \supseteq \beta_1\} \cup C_1 \cup C_2)\\
\text{where}\\
(\theta_1,C_1) &= \bigU (\hat{\tau_3},\hat{\tau_1})\\
(\theta_2,C_2) &= \bigU (\theta_1\hat{\tau_2},\theta_1\hat{\tau_4})\\
\bigU ((\hat{\tau_1} \times^{\beta_1} \hat{\tau_2}),(\hat{\tau_3} \times^{\beta_2} \hat{\tau_4})) &= (\theta_1 \circ  \theta_2, \{\beta_2 \supseteq \beta_1\} \cup C_1 \cup C_2)\\
\text{where}\\
(\theta_1,C_1) &= \bigU (\hat{\tau_1},\hat{\tau_3})\\
(\theta_2,C_2) &= \bigU (\theta_1\hat{\tau_2},\theta_1\hat{\tau_4})\\
\bigU (\hat{\tau_1} ~ list^{\beta_1}, \hat{\tau_2} ~ list^{\beta_2}) &= (\theta_1, \{\beta_2 \supseteq \beta_1\} \cup C_1)\\
\text{where}\\
(\theta_1, C_1) &= \bigU (\hat{\tau_1}, \hat{\tau_2})\\
\bigU (\hat{\tau}, \alpha) &= ([\alpha \mapsto \hat{\tau_R}], C)\text{ if }chk(\alpha, \hat{\tau})\\
\text{where}\\
(\hat{\tau_R}, C) &= repR(\hat{\tau})\\
\bigU (\alpha, \hat{\tau}) &= ([\alpha \mapsto \hat{\tau_R}], C)\text{ if }chk (\alpha, \hat{\tau})\\
\text{where}\\
(\hat{\tau_R}, C) &= rep(\hat{\tau}) 
\end{align*}

Here, $chk (\alpha, \hat{\tau})$ is a function that returns true iff $\alpha = \hat{\tau}$ or $\alpha$ is not a free variable in $\hat{\tau}$.
$rep (\hat{\tau})$ returns a type $\hat{\tau_R}$ and a substitution $C$ such that $\hat{\tau_R}$ is equal to $\hat{\tau}$, but has all annotation variables replaced by fresh ones, and the original annotation variables are constrained be greater or equal to their fresh counterparts. $repR$ does the same but has the constraints in the opposite direction.

Note that recursive calls to \bigU preserve the order of the arguments in all cases except one: The order is reversed for the input type of a function, since function inputs are contravariant.

The function \bigW performs the actual inference of annotated types. It is defined as follows:

\begin{align*}
  \bigW (\hat{\Gamma}, n) &= (Int, [], \emptyset)\\
  \bigW (\hat{\Gamma}, \textbf{true}) &= (Bool, [], \emptyset)\\
  \bigW (\hat{\Gamma}, \textbf{false}) &= (Bool, [], \emptyset)\\
  \bigW (\hat{\Gamma}, x) &= (\hat{\tau}, [], C)\\ 
  \text{where}\\
  (\hat{\tau}, C) &= instantiate (\hat{\Gamma}(x))\\
  \bigW (\hat{\Gamma}, \textbf{fn }x=> (e_0)) &= ((\theta_0 \alpha_x \xrightarrow{\beta_0} \hat{\tau_0}, \theta_0 \{\beta_0 \supseteq \{\pi\}\} \cup C_0)\\
  \text{where}\\
  \alpha_x, \beta_0 & ~are fresh\\
  (\hat{\tau_0}, \theta_0, C_0) &= \bigW (\hat{\Gamma}[x \mapsto \alpha_x], e_0)
\end{align*}


TODO

\subsection{Implementation}
The implementation of the algorithm can be found in the file \texttt{ControlFlowAnalysis.hs}. 

TODO

\subsection{Polyvariance}


\section{Evaluation of our Solution}


\section{Compiling and Running our Code}


\section{Example Programs and Results}


\end{document}
